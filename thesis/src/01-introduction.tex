% To talk:
% - Nespresso, their logistics and how can RFID help
% - grab nespresso example e estrapolar para outros markets
% - CTT, logistica, integração com amazon, how can RFID help
% - *escolas/hospitais: controlo de material disponível
% - Necessity of RFID globaly: optimizar transporte (contentores de transporte maritimo), data analisys for predictions, um tecnologia que unifica APIs de todas as empresas 
% - Benefícios de smart shelves e de que forma se encaixa no environment  

\chapter{Introduction}

\section{Background and Motivation}

The dawn of a new industrial revolution is already breathing strong. Yet, the true extension of this concept might not be fully understood.
From manufacturing to retail, the information and communication technologies disrupted the understanding of what could be improved through the \gls{supply chain} and \gls{value chain}. 
% To understand the use context of \gls{UHF RFID} we have to grasp the technologies support it and how it can empower the relationship between manufacturing, transportation, logistics and retail.

In the manufacturing process, the \gls{industry 4.0} has been the foundation of what some people call the fourth industrial revolution~\cite{marrWhatIndustryHere}.
\todo[color=cyan, inline]{A new industrial era has already started. The rise of wireless communications, and the massification of data sensing now allows for \dots}
The introduction of information and communication technologies to the manufacturing process, expanded the \gls{cps} of the third industrial revolution and introduced \gls{iot}, \gls{cloud computing} and \gls{cognitive computing}. 
It opened the door for new technologies centered around data and computation to enhance and optimize production processes, logistics operations and marketing strategies.

Adding \gls{iot} provided the network infrastructure for data transfer, \gls{cloud computing} enabled on-demand availability of \gls{IT} resources, deployment automation tools and management strategies for \gls{web services}, \gls{cognitive computing} contributed with tools to analyze and process large amounts of data for operational optimizations and pattern recognition.
These technologies support a digital end-to-end integration and optimization of the \gls{value chain} and \gls{business processes}. 

\gls{UHF RFID} improves the developments made during \emph{4.0 Industry} by extending its usability towards the \gls{supply chain}, achieving end-to-end integration of the complete commercialization process. It is a\todo[color=cyan, ]{heck} wireless technology capable of connecting billions of physical items to the internet, enabling businesses and consumers to identify, locate and engage each item.

Empowering the ability of uniquely identify items, granted \gls{UHF RFID}, with the technologies previously mentioned, provides the means for \gls{supply chain} optimization, automatic stock replenishment strategies, marketing plans, data analysis, etc. \todo[]{Improve the applications of RFID}

\todo[color=cyan,inline]{Talk about investment in RFID by companies}

The future of \gls{RFID} seems promising. In 2016, 96\% of retailers had plans to deploy tags on their apparel products~\cite{hardgrave2016StateRFID}. \gls{RFID} can greatly improve transport system with cost reductions, smaller inventory, faster transportation and routing troubleshooting, lower insurance rates and greater efficiency~\cite{oanaRFIDTechnologyContainers2013}.

Development and standardization of \gls{UHF RFID} enables a symbiotic relation across the chain, connecting production, logistics, retail and client.

\section{Scope}

\todo[color=yellow,inline]{Smart Shevels: where they stand in the RFID chain, what they bring to the market}

\todo[color=yellow,inline]{Nespresso: present conditions, how smart shelves can help (logistics, supply chain)}
\todo[color=yellow,inline]{CTT: present conditions, how smart shelves can help (internationalization: amazon partnership, logistics)}
\todo[color=yellow,inline]{Industry, Retail and Schools: present conditions, how smart shelves can help (identification of status and person responsable for equipment, control, stocks on warehouses, logistics)}


%The proposed solution is a system around \textbf{smart shelving}. 

%The structure storing the products contains RFID antennas and readers that detect and read the tags attach to them. Those readers will let the platform know in real time the state of the product in stock.

%This system should handle the registration and verification of arriving stock and manage in real time warehouse products. 

%The product should integrate with the logistics management software used by the company, allowing the a real-time management off all the products, machine learning predictions and control of the product flow.

%The solution must be reliable and cheap to maintain. The initial investment should also be the smallest possible. 

%Nespresso is owned by Nestlé Nespresso S. A., one operational unit of the Nestlé Group, with headquarters in Lausanne, Switzerland~\cite{nespressowebsite}. (...)

%With the growing of the brand, the complexity in the logistics networks starts to compromise the management of the products down in the chain. 

%The categorization and verification of new inventory, inspection of the arrived goods from the transportation company, returns, control and management of stocks, are all attended by manual labour. 
%The manual labour is prone to errors, takes a lot of working time and interfacing with the management software isn't usually efficient.

%\section{Objectives}

%\begin{itemize}
  %\item \textbf{Prevent stock-outs:} get timely replenishment and optimizes in-store sales and management. Logistics companies deliver goods on time and according to delivery requirements;
  %\item \textbf{Reduce time and errors from manual labour:} counts, identification, misplacement and lost or stolen items;
  %\item \textbf{Help customers:} find and engage with the products they want;
  %item \textbf{Control:} who removes or checks out valuable items;
  %\item \textbf{Automatic information and management of stock:} the logistics lines automatically transmits and receives stock information;
  %\item \textbf{Smart physical storage:} automatic identification of goods in the warehouse/shelves, automatic matching of distribution requirements, improves the efficiency of goods storage;
  %\item \textbf{Acquisition technology:} After the goods enter the collection area, the collection equipment automatically identifies multiple items by collecting RFID tags, thereby efficiently completing the goods in and out of the warehouse, ensuring whether the physical and distribution requirements are consistent, and improving the efficiency of goods distribution;
  %\item \textbf{Real-time:} master the distribution of all goods in real time, accurately grasp the inventory situation, optimize the reasonable inventory, and grasp the status and changes of the warehouse environment in real time;
%\end{itemize}

\section{Outline}

\todo[color=yellow,inline]{Dissertation organization}

%\cleardoublepage