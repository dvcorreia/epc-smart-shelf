% - history
% - overview da architectura (tag, antena, leitor)
% - basics da física
% - tecnologias (LF, HF), seus protocolos e aplicações
% .
% - UHF rain RFID
% - Problems (codigo de barras inoperability, diferentes standards )
% - solution of EPC (who is it) 
% - architecture solution
% - Tags: coding schemes, types ...
% - Reader and LLRP
% - Filtering & collection and ALE 

\chapter{State of the art}

\section{Basic principles of RFID}

\todo[inline]{History of RFID}
\todo[inline]{Physical overview (Induction and RF)}
\todo[inline]{Tecnologies (LF, HF, UHF, Microwave)}
\todo[inline]{RFID System (tag, antenna, reader)}
\todo[inline]{Advantages and Limitations}
\todo[inline]{Application Areas}

\section{UHF RFID System Architecture}

\todo[color=yellow,inline]{Brief expectation of what an RFID Architecture should be and requirements}

\subsection{Global Standardization}

\subsubsection{Importance}

\todo[color=yellow,inline]{The importance of a end-to-end architecture standard (enables comunication between every member in the value chain)}

\subsubsection{Efforts}

Standardization of \gls{UHF RFID} for item level tagging and \gls{supply chain}, by organizations like \gls{GS1}, provided a common language to identify, capture and share supply chain data, ensuring important information is accessible, accurate and easy to understand~\cite{anonymousStandardsGS12014}.

The first prominent adoption was by the \gls{DoD} with a policy released on July 30th 2004. The policy stated that contracts issued for material delivery would require the use of RFID tags. The policy was later extended to all commodities and commodities pallets shipped to any \gls{DoD} facility~\cite{DoDSuppliersPassive, DODReleasesFinal}.

In 2014, Impinj, Intel, Google and Smartrac, joined forces to create the \gls{RAIN RFID} alliance~\cite{TechnologyCompaniesCreate} after the ratification of \gls{GS1}'s \gls{UHF RFID} Generation 2 version 2 standard in November of 2013. The alliance promotes the universal adoption of \gls{GS1}'s Gen2 \gls{UHF RFID} technologies and the connection with \gls{cloud computing}, where RFID-based data can be stored, managed and shared via the Internet~\cite{WhatRAINRFID}.
The alliance fortified the adoption of \gls{GS1}'s standards and traced a common path for the the industry to progress.

On October 11th 2018 the European Commission published their positive implementing of the upper band in Europe (ETSI EN 302 2018-1 v3.1.1).
\gls{RFID} equipment operating in the band $865MHz$ to $868MHz$ with power levels up to $2W$ and in the band $915MHz$ to $921MHz$ with power levels up to $4W$.


\subsubsection{Current Problems}

\todo[inline]{International Standardization in UHF band RFID: problems (conformity, different standards), EU UHF RFID Upper band}

\subsection{GS1}

The GS1 is a nonprofit organization dedicated to the development and implementation of standards for global \gls{supply chain} solutions. 
The institution mission is to manage the GS1 System of Standards, create open, global, multi-sector standards fostering good business practices.

GS1 established itself in 2005 from the \gls{EAN} International, \gls{UCC} and other local organizations from the United States~\cite{PublicationLEBENSMITTELZEITUNGa}.
The organization took under its umbrella the former EAN-UCC roles subsuming their technologies. From those, worth mentioning: the barcode identification system (from \gls{EAN}), \gls{XML} standards, \gls{EDI} transaction sets and \gls{supply chain} solutions~\cite[p.~212]{lahiriRFIDSourcebook2005}.

The new GS1 organization then adopted much more ambitious projects, developing global standards and services for business communication.
From those efforts resulted the network for the synchronization of master data \gls{GDSN}, the \gls{EPC} integration for \gls{RFID}, traceability and the upstream integration of the consumer goods industry suppliers and EPCglobal Network.

\subsection{Architecture Framework}

\todo[color=yellow,inline]{GS1 Architecture: organization, advantages and objectives: barcode inoperability}

\subsubsection{EPCglobal Specification} \label{epcglobal}

\todo[color=yellow,inline]{Classes, codding schemes: 1 Gen 2 or ISO/IEC18000-6 Type C: ISO adoption of GS1's C1G2}

\subsubsection{Low Level Reader Protocol (LLRP)}

\todo[color=yellow,inline]{Low Level Reader Protocol (LLRP)}

\subsubsection{Middleware}

\todo[color=yellow,inline]{Middleware}

\subsubsection{Application Level Events (ALE)}

\todo[color=yellow,inline]{Application Level Events (ALE)}

\subsubsection{All working together}

\todo[color=yellow,inline]{All working together (brief on the everything working together)}

\section{Overview of the overall solutions}

\subsection{Commercial solutions}

\todo[color=yellow,inline]{Analisys of commercial antenna/reader solutions}

\subsection{Academic solutions}

\todo[inline]{Work of Andrea D’Alessandro~\cite{dalessandroRFIDBasedSmartShelving2012}}
\todo[inline]{Robots for warehouse scan}